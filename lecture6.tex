\section{Lecture 6}

\subsection{Asymetric key cryptography}

Knapsack problem/knapsack algortithms \\

A trapdoor one-way function is a function that is easy to compute but
computionally hard to reverse.\\

Extended Euclidean method\\

Division mod $n$, when you cant divide you either has no solution or many
solutions.\\

Euler's totient function \(\phi(n)\)\\

Chinese remaindering theorem\\

RSA (Rivest Shamir Adleman 1977)\\

You want the exponentiation to be cheap so you ensure that a small number of
bits is set (binary).\\

Fermat's little theorem, for choosing prime.\\

Miller-Rabin primality test, for testing prime number for high probability.\\

Miller-Rabin, Fermat test, Solovay-Strassen test are pobabilistic tests. i.e.
they can be wrong. \\

Miller-Rabin is right \(3/4\)\\

Factorization, use conjugat rule and factors of different length is prefered.\\


