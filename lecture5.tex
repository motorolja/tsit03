\section{Lecture 5}

\textbf{AES:}\\
\begin{itemize}
\item Whitening - adding the roundkey (read)
\item \(x^{-1}\) is hard to estimate with a linearsystem.
\item Finite fields (Galois Fields, GF)
\item Can be done with binary muliplications and additions
\item Has protection against partial knowledge of the key
\item AES has been broken but not terrible broken
\end{itemize}

\textbf{Blowfish}, is broken since it is a 64 bit block cipher.

\subsection{Modes of operation}
\begin{itemize}
\item{ECB (Electronic Code Book):
  }

\item{CBC (Cipher Block Chaining):
    \begin{itemize}
    \item Sees if there are noise or tampering since the decryption will fail.
    \item Dont use a fixed initialization vector (IV)
    \end{itemize}
  }
\item{CFB (Cipher FeedBack):
    
  }

\item{OFB (Output FeedBack)
    
  }
\item{CTR (Counter Mode):

  }

  
\end{itemize}

\subsection{Message Authentication Code}
\begin{itemize}
  \item{An ideal MAC should behave as a random mapping from all possible inputs
      to all possible tags.}
    \item{This requires using a secret key.}
\item{CBC-MAC (CBC message authentication code):
    \begin{itemize}
    \item{Dont use the same key for encryption and authentication}
    \item{Birthday attack on CBC-MAC:
        See sweet32.info
      }
    \item{Length attack on CBC-MAC:
        See slide lecture 5
      }
    \end{itemize}

  }
\item{CMAC:
    

  }
\item{The Horton Principle:
    \begin{itemize}
      \item{Authenticate what is meant, not what is said}
     \end{itemize}
   }
\item{AES-CGM is broken, check blackhat article about it from 2016 Sean Devlin
    and Hannon B\"{o}ck}
\end{itemize}
